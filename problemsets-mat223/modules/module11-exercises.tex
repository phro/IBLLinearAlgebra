\begin{exercises}
	\begin{problist}
		\prob For the following matrices, find their null space, column space,
		and row space.
		\begin{enumerate}
			\item $M_{1}=\mat{1&2&1\\3&1&-2\\8&6&-2}$.

			\item $M_{2}=\mat{0&2&1\\3&2&5}$.

			\item $M_{3}=\mat{1&2\\3&1\\4&0}$.

			\item $M_{4}=\mat{1&-2&0&-1\\3&5&-1&0\\2&3&-2&0\\0&0&0&1}$.
		\end{enumerate}

		\prob Let $\mathcal P$ be the plane given by $3x+4y+5z=0$, and
		let $T:\R^{3}\to\R^{3}$ be projection onto $\mathcal P$.
		\begin{enumerate}
			\item Find $\Range(T)$ and $\Rank(T)$.

			\item Find $\Null(T)$ and $\Nullity(T)$.
		\end{enumerate}

		\prob Find the range and null space of the following linear
		transformations.
		\begin{enumerate}
			\item $\mathcal P:\R^{2}\to\R^{2}$, where
				$\mathcal P$ is projection on to the line
				$y=x$.

			\item Let $\theta\in \R$ and let $\mathcal R:\R^{2}\to\R^{2}$
				to be the transformation which rotates all vectors
				by counter-clockwise by $\theta$ radians.

			\item $\mathcal F:\R^{2}\to\R^{2}$, where
				$\mathcal F$ reflects over the $x$-axis.

			\item $\mathcal M:\R^3\to\R^3$ where $\mathcal M$ is the matrix
				transformation given by $\mat{1&2&3\\4&5&6\\7&8&9}$.
			\item $\mathcal Q:\R^3\to\R^1$ defined by $\mathcal Q\mat{x\\y\\z}=x+z$.
		\end{enumerate}

		\prob 
		\begin{enumerate}
			\item Let $\mathcal T$ be the transformation induced by the
				matrix $\mat{7&5\\-2&-2}$, and
				$\vec v=3\xhat-3\yhat$.
				Compute $\mathcal T\vec v$ and $[\mathcal T\vec v]_{\mathcal E}$.

			\item Let $\mathcal T$ be the transformation induced by the
				matrix $\mat{3&7&5\\1&-2&-2}$, and
				$\vec v=2\xhat+0\yhat+4\zhat$.
				Compute $\mathcal T\vec v$ and $[\mathcal T\vec v]_{\mathcal E}$.
		\end{enumerate}

		\prob For each statement below, determine whether it is true or false. Justify you answer.
		\begin{enumerate}
			\item Let $A$ be an arbitrary matrix. Then $\Range(A)=\Range(A^{T})$.

			\item Let $T:\R^{m}\to\R^{n}$ be a transformation (not necessarily
				linear). If $\Null(T)=\Set{x \given T(x)=0}$
				is a subspace, then $T$ is linear.

			\item Let $T:\R^{m}\to\R^{n}$ be a linear transformation.
				Then $\Nullity(T) \geq n$.

			\item Let $T:\R^{m}\to\R^{n}$ be a linear transformation
				induced by a matrix $M$. If
				$\Rank(T) = n$, then $\Nullity(M) = 0$.
		\end{enumerate}
	\end{problist}
\end{exercises}
